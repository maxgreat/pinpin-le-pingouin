\documentclass{article}
\usepackage[utf8x]{inputenc}
\usepackage[T1]{fontenc}
\usepackage[frenchb]{babel}
\usepackage{lmodern}
\usepackage{graphicx}
\usepackage[margin=1in]{geometry}
\usepackage{fancyhdr}
\usepackage{lastpage}
\usepackage{titling}
\usepackage{MnSymbol}
\usepackage{listingsutf8}


\lstset{breaklines=true, breakatwhitespace=false}
\lstset{frame=single}
\lstset{postbreak=\raisebox{0ex}[0ex][0ex]{
    \ensuremath{\rcurvearrowse\space}}}
\lstset{language=C}

\pagestyle{fancy}
\fancyhead{}
\renewcommand{\headrulewidth}{0pt}
\fancyfoot{}
\renewcommand{\footrulewidth}{0.4pt}

\fancyfoot[C]{\thepage/\pageref{LastPage}}
\fancyfoot[L]{\textit{\theauthor}}
\fancyfoot[R]{\textit{\today
}}

\let\maketitlebis\maketitle
\renewcommand\maketitle{\maketitlebis \thispagestyle{fancy}}


%% Informations
\title{Présentation PROG6}
\author{Guiroux Hugo}

\begin{document}

\maketitle

\section{Objectif :}
Réaliser un logiciel de taille moyenne en groupe de 6.
\textbf{Nouveautés : } 

\begin{itemize}
\item taille du logiciel
\item gestion de l'équipe
\end{itemize}

Le logiciel sera : \textbf{jeu de type réflexion à deux joueurs avec jeu de l'ordinateur et IHM.} (choix mercredi)
\\
\section{Evaluation :}
Soutenance + démo à la fin du projet
\\
\textbf{Critères} :
\begin{itemize}
\item Jeu de l'ordinateur :
  \begin{itemize}
  \item pertinence de l'algorithme
  \item tests de validation
  \item présentation
  \end{itemize}
\item IHM :
  \begin{itemize}
  \item qualité générale du résultat
  \item progrès au fil du projet
  \end{itemize}
\item Qualité technique générale
\end{itemize}

\newpage
\section{Planning}
\begin{itemize}
\item Semaine 1 :
  \begin{itemize}
  \item Pré-projet => mercredi (10h) : mini-démo
  \item Mercredi aprem : 
    \begin{itemize}
    \item amphi d'IHM
    \item amphi jeu \textbf{ => tuteur assigné}
    \end{itemize}
  \item vendredi 9h45 : \textbf{amphi réseau}
  \end{itemize}
\item Semaine 2 :
  \begin{itemize}
  \item Mardi 10h - ... : audit IHM \textbf{ : préparer une maquette papier}
  \item Vendredi 8h... : audit I.A \textbf{ : préparer un algorithme pour le jeu de l'ordi}
  \end{itemize}
\item Semaine 3 :
  \begin{itemize}
  \item Jeudi 8h... : audit IHM \textbf{ : logiciel}
  \end{itemize}
\item Semaine 4 :
  \begin{itemize}
  \item Lundi 9h45 : amphi soutenances 
  \item et ensuite (deux vidéo projecteurs disponibles : (présentation + démo)
    \begin{itemize}
    \item préparer la démo
    \item répeter la soutenance
    \end{itemize}
  \end{itemize}
\item Semaine 5 : 
  \begin{itemize}
  \item Lundi-Mardi : 10-11
    \begin{itemize}
    \item Soutenances : 1h par groupe
    \item 30 min de présentation + 30 min de questions
    \end{itemize}
  \end{itemize}
\end{itemize}
\newpage
\section{Rendu}
\textbf{Documents + logiciels à fournir : envoyer une archive à guillaume.huard@imag.fr}
\begin{itemize}
\item pré-projet mercredi (logiciel) => \textbf{envoyer une fois terminé}
\item maquette papier I.H.M. (ne pas envoyer)
\item lors de la soutenance :
  \begin{itemize}
  \item manuel utilisateur
  \item dossier de validation du joueur ordi :
    \begin{itemize}
    \item explication de l'algorithme
    \item résultats de la ou des campagne(s) d'évaluation  (multiples version de l'algo)
      \begin{itemize}
      \item validation des niveaux 
      \item non régression des différentes versions
      \item validation contre I.A d'autres groupes
      \end{itemize}
    \end{itemize}
  \item logiciel final
  \end{itemize}
\end{itemize}
\newpage
\section{Pré-Projet : la gaufre empoisonnée}
Gaufre modélisée par une grille de n x m.\\
Coin inférieur gauche contient du poison.\\
Chacun leur tour les deux joueurs doivent manger un bout de gauffre en sélectionnant une des cases de la gauffre (en mangeant le quart de plan supérieur droit via à la case sélectionnée).\\
Le joueur qui doit manger le poison a perdu.\\\\
Ce qui est demandé : \\
\begin{itemize}
\item une interface utilisateur bien pensée
  \begin{itemize}
  \item interaction pratique (clic à la souris)
  \item lisibilité de l'interface
  \item toute l'information utile doit être présentée :
    \begin{itemize}
    \item A quel joueur ?
    \item Nombre de coup jouer ?
    \item Historique ?
    \item Position du poinson
    \item Quel joueur gagne
    \end{itemize}
  \item Eviter à l'utilisateur de cliquer sans arrêt ou de bouger la souris si ce n'est pas nécessaire
  \end{itemize}
\item Jeu à deux joueurs (sans I.A)
\item jeu à l'ordinateur 
  \begin{itemize}
  \item jeu aléatoire
  \item joue les coups gagnants
  \item minmax complet
  \item minmax seuil
  \end{itemize}
\item Fonctionnalités :
  \begin{itemize}
  \item nouvelle partie
  \item sauvegarder (avec historique)
  \item charger
  \item annuler/refaire sur un nombre de coup arbitraire
  \item abandonner
  \end{itemize}
\item Extensions : (optionnel) : 
  \begin{itemize}
  \item mode match (enchainer les parties)
  \item coup conseillé par l'ordinateur
  \item tutoriel
  \item jeu en réseau
  \item animation
  \item scoreboard
  \end{itemize}  
\end{itemize}
\newpage
\section{Organisation du code :}
Découper en parties distinctes :\\ 
Exemples :\\
\begin{itemize}
\item Interface : textuelle ou graphique, isolée de l'applicatif
\item Moteur de jeu (arbitre) : vérifie la validité des coups, implémente la sauvegarde restauration
\item Joueur : demande coup possibles à l'arbitre et applique le minmax pour choisir le coup (implémentation de l'IA)
\end{itemize}

\section{Organisation du travail, rôles :}
\begin{itemize}
\item rapporteur : tenue d'un journal détaillant l'évollution du projet.
  \begin{itemize}
  \item aide à construire la présentation
  \item aide à ne pas reproduire plusieurs fios les mêmes erreurs
  \item permet d'interréagir avec le tuteur
  \end{itemize}
\item testeur : teste les différentes versions du projet
  \begin{itemize}
  \item non-regression
  \item identification de versions stables
  \end{itemize}
\end{itemize}
\item intégrateur : met ensemble les différentes parties du code
  \begin{itemize}
  \item cohérence globale
  \item identification de problèmes d'architecture
  \end{itemize}
\item encadrement / suivi :
  \begin{itemize}
  \item durant la gaufre : passage sporadique d'enseignants dans les salles.
  \item ensuite : 
    \begin{itemize}
    \item contacter son tuteur
    \item lui fournir les infos de suivi (journal)
    \item prendre des rdv avec lui
    \end{itemize}
  \end{itemize}
\item hotline : \textbf{concombre.masque@imag.fr}
\end{itemize}
\end{document}
