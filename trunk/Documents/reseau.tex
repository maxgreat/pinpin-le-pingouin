\documentclass{article}
\usepackage[utf8x]{inputenc}
\usepackage[T1]{fontenc}
\usepackage[frenchb]{babel}
\usepackage{lmodern}
\usepackage{graphicx}
\usepackage[margin=1in]{geometry}
\usepackage{fancyhdr}
\usepackage{lastpage}
\usepackage{titling}
\usepackage{MnSymbol}
\usepackage{listingsutf8}


\lstset{breaklines=true, breakatwhitespace=false}
\lstset{frame=single}
\lstset{postbreak=\raisebox{0ex}[0ex][0ex]{
    \ensuremath{\rcurvearrowse\space}}}
\lstset{language=C}

\pagestyle{fancy}
\fancyhead{}
\renewcommand{\headrulewidth}{0pt}
\fancyfoot{}
\renewcommand{\footrulewidth}{0.4pt}

\fancyfoot[C]{\thepage/\pageref{LastPage}}
\fancyfoot[L]{\textit{\theauthor}}
\fancyfoot[R]{\textit{\today
}}

\let\maketitlebis\maketitle
\renewcommand\maketitle{\maketitlebis \thispagestyle{fancy}}


%% Informations
\title{Projet PROG6 : Réseau}
\author{Guiroux Hugo}
\begin{document}
\maketitle

\section{Plan}
\begin{itemize}
\item Architecture de l'application répartie
\item Protocole d'interaction
\item Messages
\item programmation avec sockets TCP/IP en Java
\end{itemize}

\section{Architecture de l'application répartie}
Deux modèles : \textbf{liaison directe entre deux machines} et \textbf{serveur central + clients}.
\section{Liaison directe}
\textbf{Deux instances du jeu, un joueur par instance}\\
Eventuellement adaptable à N joueurs par instances mais plus compliqué.\\
\textbf{Dans ce schéma, chaque instance exécute à la fois l'interface et le moteur de jeu}\\
L'état du jeu est dupliqué et doit rester cohérent\\
\textbf{Problème à considérer}\\
\subsection{Démarrage}
Les deux instances jouent un rôle symétrique.\\
Mais du point de vue de TCP, il faut nécessairement un client et un serveur\\
Le rôle de client ou serveur au niveau TCP n'a pas nécessairement d'impact sur les caractéristiques applicatives (ordre de passage, rôle dans le jeu, ...)\\
\subsection{Mélange des interactions GUI et réseau}
La GUI peut être gelée en attente d'un message de l'autre joueur.\\
Si le (seul) thread de la GUI effectue une lecteur bloquante sur la socket de dialogue.\\
Pise de solution\\
Découpage des activités en utilisant plusieurs flots d'éxécution (ici, des threads).\\
\begin{itemize}
\item 1 thread GUI
\item 1 ou 2 thread réseau (émission/réception)
\item 1 thread moteur
\end{itemize}
Schéma producteur-consommateur :
\begin{itemize}
\item Chaque thread dispose d'une file d'attente (boite aux lettres) dans laquelle on peut lui poster des messages.
\item Chaque thread traite ses messages un par un ... et un traitement peut produire un ou plusieurs messages à destination des autres threads
\item Pour le therad réception réseau, la file correspond à la socket de dialogue
\item Pour le thread gui ...
\end{itemize}

\section{Serveur central}
N instances (1 joueur par instance) + 1 serveur central\\
Eventuelement adaptable à M joueurs par instance mais plus compliqué.\\
Dans ce schéma\\
Le serveur exécute le moteur de jeu et gère son état.\\
Les clients exécutent uniquement l'interface du jeu.\\
\\
Problème sà considérer\\
Mélange des interactions GUI et réseau
\\ Propagation des mises à jour liées aux actions d'un client.\\
Vers tous les autres clients.\\
\\
Pises de solutions :\\
Client : cf piste proposée pour le premier modèle\\
Mais pas de véritable thread moteur. Seulement une simple conversion (bidirectionnelle) entre événements réseau et événements GUI.\\\\
Serveur centra l
\begin{itemize}
\item 1 thread pour le moteur
\item 1 ou 2 thread de dialogue par client
\item Schéma producteur-consommateur(s)
\begin{itemize}
\item 1 file dans le sens client <-> serveur
\item N files (1 par client) dans un sens serveur clients
\end{itemize}
\end{itemize}

\section{Serveur salon de jeu}
Quelle que soit l'architecture retenue, on peut éventuellement utiliser un serveur faisant office de salon de jeu\\
Intérêt: \\
Point de rencontre / annuaire des utilisateurs (dont les machines ou adresse IP peuvent changer d'une fois sur l'autre)\\
Authentification des utilisateurs.\\
Mémorisation des scores, etc.
\\
\\
Pour le 1er modèle : 
Le serveur salon est seulement un point de rendez-vous. Il n'est pas impliqué au cours d'nue partie. \\
Le serveur salon créé une instance de serveur central pour chaque partie en cours.

\section{Protocol d'interaction}
\subsection{L'application répartie est composée d'un ensemble d'instances qui interagissent}
Il faut spécifier clairement les séquences d'interactions possibles.\\
Sous la forme d'une ou plusieurs machines à états :
En particulier, dans un état donné, quels sont les transitions possibles
\begin{itemize}
\item Réception d'un message ? Et le cas échant, quel message ?
\item Action de l'utilisateur local ? Et le cas échant, quelles actions ?
\item Les deux types d'événements
\end{itemize}
Bien vérifier que la spécification du protocole est correcte, en considérant les critères suivants :
\begin{itemize}
  \item \textbf{Sûreté : Est-ce que l'état global de l'application est toujours cohérent ?}
    Exemple : Des instances peuvent elle avoir une vue divergente de la séquence de coups déjà joués ? Ou du nombre de pions/cartes/points dont dispose chaque joueur ?
  \item Progrès : L'application peut-elle toujours avancer (en supposant que les joueurs répondent en temps borné) ?
    \begin{itemize}
      \item Peut-on tomber dans une situation où les différentes instances s'attendent mutuellement ?
      \item Peut-on tomber dan sun dialogue sans fin entre plusieurs instances (non lié à l'attitude des joueurs) ?
    \end{itemize}
\end{itemize}

Gestion des cas d'erreur et de terminaison.\\
Parmi les cas d'erreurs possibles quels sont ceux :\\
qui nécessitent seulement un traitement local ?\\
Qui nécessite d'informer une ou plusieurs des autres instances ?
\\
\\
A quelles phases de l'application un utilisateur peut-il demander sa terminaison ?\\
\\
Comment/où est gérée la déconnexion brutale d'un utilisateur ?\\
Détection ?\\
Réaction ?\\
Il vaut mieux empécher le problème dans le protocol (évite annulation de coup => frustration).\\\\
\section{Messages}
Rappels:\\
Un canal TCP est un flux d'octets \\
En réception, l'application doit lire et découper ce flux en une séquence de messages.\\
Le protocol doit définir:\\
\begin{itemize}
\item Les types de emssages utilisés
\item Le format de chaque type de message
\end{itemize}

Selon les besoins on peut utiliser
\begin{itemize}
\item Des messages de taille fixe (peut êtr econtraignant)
\item Des messages de taille variable
\item Nécessitent un en-tête de taille fixe et/ou des marqueus
\end{itemize}

Transmission des arguments d'un message :\\
Par valeur\\
Pas de difficulté pour les types de base\\
Nécessité de "sérialiser" les structures de données plus complexes. Deux aspects :
Transformer une représentation spatiale (avec des liens d'indirection) en une séquence d'octets.\\
Obtenir une représentation auto-contenue, c-a-d sans références (adresses mémoire) qui n'ont de sens que localement.\\\\
Attention à bien découpler :\\
La manipulation d'un message applicatif.\\
Et l'interaction avec une socket (écrire un message dans une socket avec sérialisation préalable si nécessaire). Lire un message.

\textbf{Les différentes étapes de lecture (ou d'écrire) d'un message dansune socket ne doivent pas être éparpillées à différents endroits du code}.
\end{document}
